\documentclass[titlepage]{article}
\usepackage{graphicx}
\usepackage{amsmath}
\usepackage[T1]{fontenc}  
\usepackage[utf8]{inputenc}   
\usepackage{geometry}
\usepackage{caption}
\usepackage{hyperref}
\usepackage{tabularx,ragged2e,booktabs,caption}
\newcolumntype{C}[1]{>{\Centering}m{#1}}
\renewcommand\tabularxcolumn[1]{C{#1}}
\usepackage[font=small,labelfont=bf]{caption}
 \geometry{
 a4paper,
 total={210mm,297mm},
 left=20mm,
 right=20mm,
 top=20mm,
 bottom=20mm,
 } 
\usepackage{eso-pic}
\AddToShipoutPicture{%
  \AtPageUpperLeft{%
    \hspace*{20pt}\makebox(200,-20)[lt]{%
      \footnotesize%
      \textbf{Lucas Hellström}%
}}}




\title{Using TTV signals to catalogue multi-planet systems}
\author{Lucas Hellström - 950905-0655}
\date{}

\begin{document}
\maketitle

\section{Introduction}
	By measuring Transit Timing Variations, TTVs, it is possible to separate a single-planet system from a multi-planet system. This is done by measuring the time the planet takes to make one complete revolution around its host star. If the time for one revolution differs by a non-negligible amount the system can be further observed to confirm the existents of multiple planets around the star. 
	
	This paper will study transit times measured by the Keppler satellite and, by specifying a number of parameters, simulate the data that the upcoming TESS satellite will produce. This data is analysed to get a better understanding of multi-planets systems and to create a catalogue of these systems for further study. 
\section{Theory}
	The easiest obtainable information about a star far way is the intensity of the star. By measuring this over some time a light curve can be made which describes the intensity of the star over some time. If there is a decrease in the intensity in regular intervals a planet might be transiting. A transit is a phenomenon where a planet moves between a star and the observer causing the intensity to decrease by blocking some of the light emitted by the star. This can be seen in figure \ref{fig:trans}
	
	\begin{figure}[h!]
		\centering
		\captionsetup{justification=centering}
		\includegraphics[width=0.6\textwidth]{Planetary_transit.svg}
		\caption{Illustration of a transiting planet and the resulting light curve. \\  \small{ Source: \url{https://commons.wikimedia.org/wiki/File:Planetary_transit.svg}}}
		\label{fig:trans}
	\end{figure}
	, TTV, TTVFast, Sullivan, Keppler, TESS
\section{Method}
	Coding
\section{Results}
	Catalogue
\section{Discussion}
	Important?

\end{document}

