%%
%%  This is a LaTeX template for an astronomy Bachelor's thesis.
%%
%%  Version 1.0
%%
%%  Authors: Anders Johansen, Sofia Feltzing, Johan Bijnens (2014)
%%  Send feedback to Anders Johansen <anders@astro.lu.se>
%%
\documentclass[12pt]{report}

\usepackage{a4wide}
\usepackage{graphicx}
\usepackage{natbib}
\usepackage[T1]{fontenc}  
\usepackage[utf8]{inputenc} 
\usepackage{geometry}
\usepackage[justification=centering]{caption}
\usepackage{subcaption}

\usepackage{fancyhdr}
\usepackage{lastpage}
\usepackage{pdfpages}
\usepackage{url}

\pagestyle{fancy}
%\rhead{}
%\chead{}
%\lhead{}
%\rfoot{}
\cfoot{\thepage}
%\lfoot{}

\newcommand{\apj}{ApJ}
\newcommand{\apjs}{ApJS}
\newcommand{\apjl}{ApJ}
\newcommand{\mnras}{MNRAS}
\newcommand{\aap}{A\&A}
\newcommand{\aj}{AJ}
\newcommand{\nat}{Nature}
\newcommand{\pre}{Phys.~Rev.~E}
\newcommand{\araa}{ARA\&A}
\newcommand{\icarus}{Icarus}
\newcommand{\procspie}{Proceedings of the SPIE}

\begin{document}

\title{\huge \bf Multi-planetary Systems from Simulated TESS Transit Timing Variations\footnote{This is just a very basic cover page produced by LaTeX -- when
the thesis is done you can get a more formal cover page from Eva Jurlander.}}
\author{Lucas Hellström}

\thispagestyle{empty} % do not count pages just yet

\maketitle

\newpage

\thispagestyle{empty}

\begin{center}
  (this page will contain some more official information in the final version)
\end{center}

\newpage

\thispagestyle{empty}

\begin{center}
  {\bf Abstract}
\end{center}
	A transit is a phenomenon where a planet passes between its host star and an observer blocking out part of the light from the star. This decrease can be measured and used to gain information of the planet. The Kepler and TESS telescopes are examples of space telescopes using the transit method to detect exoplanets. For systems with more than one planet around the same star variations between the time a planet takes to transit might occur. These variations, called Transit Timing Variations, or TTVs, can be used to gain information about additional planets in the system.
	
	This paper uses data from \cite{2015ApJ...809...77S} and Kepler data from the NASA Exoplanet Archive to create artificial systems like those that TESS might observe. These systems are simulated using TTVFast \citep{2014ApJ...787..132D} to obtain TTV signals which are then used to create a sky map to show the fraction of systems showing TTV signals in a given sample.
	
	TESS have coverage of the ecliptic poles for a whole year which results in that many systems showing considerable TTV signals are located at the poles although there are systems located outside the poles still showing TTV signals. In the range of declination from about $-40^{\circ}$ to $40^{\circ}$ CHEOPS will be able to further study the objects. From the results of this paper it is expected that CHEOPS can find a TTV signal from about every fourth transiting planet. 
	
	
\newpage

\thispagestyle{empty}
\mbox{} % this is how we create an empty page in LaTeX

\newpage

\thispagestyle{empty}

\begin{center}
  {\bf Popul\"arvetenskaplig beskrivning}
\end{center}
	När vi letar efter exoplaneter finns det ett antal olika metoder för att hitta dem. Den mest framgångsrika är transitmetoden där ljusstyrkan hos en stjärna studeras under en längre tid. När en planet passerar mellan sin stjärna och en observatör kan en minsking i stjärnans ljusstyrka ses. Uppreras detta i regelbunda intervall kan slutsatsen att det finns en planet runt stjärnan dras. Genom att studera minskningen i ljusstyrka kan storleken på planeten beräknas vilket kombinerat med massan som fås av andra metoder ge en insikt i hur och vad planeten är uppbygd av. En transit är detta fenomen då en planet passerar mellan stjärnan och en observatör.
	
	Genom att jämföra tiden mellan varje transit för en planet kan ibland variationer ses, vilket kallas Transit Timing Variations eller förkortat TTV. Detta beror på att det finns fler planeter runt stjärnan som med hjälp av gravitationskraften accelererar eller decelerera planeten som bevakas. Detta resulterar i att det är möjligt att hitta planeter som genom andra metoder är osynliga. 
	
	Keplerteleskopet är ett rymdbaserat teleskop som använder transitmetoden för att hitta exoplaneter. Det har sedan 2009 hittat över 1000 bekräftade exoplaneter vilket gör den till det hittils mest framgångsfulla uppdraget i jakten på exoplaneter. TESS, vilket står för Transiting-Exoplanet Survey Satellite, är ett teleskop som sköts upp den 18de april 2018 och använder transitmetoden för att hitta exoplaneter. TESS kommer bli det första rymdbaserade teleskopet att studera hela himlen och kommer observera över 200 000 stjärnor under uppdragets urspungliga längd på två år.
	
	Detta projekt kommer använda data från Keplerteleskopet för att simulera data från TESS för att sedan använda den datan för att leta efter TTV signaler. Detta ska ge en uppfattning om hur många system som har fler än en planet inom ett givet område på himlen.
	
	


\newpage

\thispagestyle{empty}
\mbox{} % make sure that TOC starts on a right page

\newpage

\setcounter{page}{1} % start counting pages

\tableofcontents

\newpage

\listoffigures 
\listoftables

\newpage

\chapter{Introduction}
When observing stars in the search for exoplanets a few different methods can be used. The most successful method so far is the transit method which measures the brightness of a star for a long period. If a planet passes between the star and the observer it will block out part of the light and the brightness will decrease. If these decreases occur at regular intervals the conclusion that the reason for this is an exoplanet can be drawn. By studying the amount of light the planet blocks out the radius of the planet can be found, combined with the mass of the planet obtained from different methods the approximate density of the planet can be calculated. This gives information about the structure and material of the planet.

In a system with multiple planets around the same star, the plants will affect each other through their gravitational pull. This results in the planets accelerating or decelerating depending on the relative positions of the planets. As this is happening the time of one orbit may differ and by studying these variations planets which may not be possible to detect through the transit method can be detected.

This paper will simulate data from the TESS telescope to search for these transit timing variations in order to determine the approximate fraction of multi-planetary systems in a given sample.

\section{Transits}
	A planet in orbit around its host star may sometimes cross the line of sight of an observer. When this happens a slight decrease in the star's brightness can be measured. This is called a transit and is today used as a main method to discover exoplanets. From transits the radius of the planet can be determined but it can also be used to find additional planets around the host star which may not be transiting. This will be discussed in section \ref{sec:trans_vari}. With the radius known from the transit method and the mass obtained from different methods such as, for example the radial velocity method, the density of the planet can be calculated. The density is important to understand what the planet is made of and the structure of it.

\subsection{Variations}
\label{sec:trans_vari}

	For a system with a single planet around a star the period of the planet is more or less perfectly periodic with no visible variations but, when measuring the time of between a planet's transits one may discover variations in the period which are called Transit-Timing Variations or TTVs. These variations arise from another planet in the system whose gravitational pull accelerates or decelerates the observed planet which results in increased or decreased transit times. An advantage of studying transits in search for TTVs is that planets which does not transit their star can be discovered through TTVs \citep{0004-637X-777-1-3}. As most planets does not transit their star this can increase the number of known exoplanets drastically.

\section{Kepler}
	The Kepler satellite launched in spring 2009 on a mission to study stars in a small patch in the sky to discover Earth-sized exoplanets within the habitable zone, where liquid water can exist on the planetary surface. The brightness of a large amount of stars are measured and then analyzed in order to detect transiting exoplanets. 
	
	Kepler started by looking at a very small patch of the sky but in July 2012 one of the four wheels used to keep the patch in focus broke. The telescope requires at least three wheels to function which kept the mission alive. In May 2013 a third wheel failed which resulted in the telescope no longer being able to collect data. The satellite was nonfunctional until the so-called "Second Light (K2)" in early 2014. This mission would use the telescopes remaining two wheels to study stars over a much larger area but for shorter periods. \citep{2017PAPhS.161...38B}
	
	Kepler have found over 4500 exoplanet candidates \citep{2017PAPhS.161...38B} to this day. This amount makes Kepler the most successful exoplanet hunting mission to this date.
	
	
\section{TESS}
	The Transiting Exoplanet Survey Satellite, TESS, is a satellite which were launched on the 18th of April 2018. The satellite is equipped with four cameras which will study the brightness of over 200 000 stars over a two year period. It is the first all-sky transit survey taking place in space. \citep{2014SPIE.9143E..20R}
	
	TESS will study the whole sky by splitting it into 26 sectors, 13 in the southern hemisphere and 13 in the northern hemisphere, which are observed for 27 days each. An illustration of this can be seen in figure \ref{fig:tess_time} where the number of times TESS will observe each sector is shown. TESS will observe the southern hemisphere during the mission's first year, it will then rotate and observe the norther hemisphere for another year.
	\begin{figure}[h!]
	\centering
		\includegraphics[width=0.8\textwidth]{img/tess_observe_time.png}
		\caption{Illustration of the number of times TESS will observe each sector in the sky.\\ \small{Source: \cite{2015ApJ...809...77S}}}
		\label{fig:tess_time}
	\end{figure}	
	
\chapter{Method}
	This paper uses results from \cite{2015ApJ...809...77S} where potential system which TESS might find are simulated. The Sullivan et al. paper contains only one planet per system and therefore these results are matched together with Kepler data to obtain artificial systems. These systems are then simulated to obtain TTVs. The steps taken are listed below:
	
\begin{enumerate}
	\item For each planet in the Sullivan et al. catalogue a similar planet within 10\% in radius and period are found in the Kepler dataset.
	\item The system are then filled with additional planets taken from the Kepler archive and modified based on the ratio of the radius and period of the first Sullivan and Kepler planet.
	\item These planets are then assigned a mass depending on the radius of the planet \citep{2015ApJ...809...77S} and an eccentricity based on a Rayleigh distribution as none of those parameters are obtained from the light curves.
	\item Real systems are simulated and the results compared to those of \cite{2018ApJS..234....9O} to verify that the methods and assumptions give realistic results.
	\item The systems are then simulated using TTVFast \citep{2014ApJ...787..132D} and the resulting transit times are analyzed. First a linear fit needs to be applied and subtracted to remove the period of the planet and thus obtain the variation. The amplitude of the variation is obtained from average of the highest and lowest value.
	\item The error of the results are calculated from \cite{2005Sci...307.1288H} and is dependent on the magnitude of the host star, the ratio of radius between the planet and host star and the transit duration.
	\item To ensure that the systems are realistic they are simulated for stability over a long time. This is done using WHFast \citep{2015MNRAS.452..376R} and IAS15 \citep{2015MNRAS.446.1424R}.
\end{enumerate}

\section{Simulation of TESS objects}
\label{simTESS}
	\cite{2015ApJ...809...77S} provides one planet per system, by combining data from Kepler obtained from the NASA Exoplanet Archive\footnote{\url{https://exoplanetarchive.ipac.caltech.edu/index.html}} and the results from Sullivan et al., TESS data are simulated. The planets are separated into two groups based on the effective temperature of their host star to prevent planets around cold stars to get "matched" with planets around hot stars. Planets around stars with an effective temperature below 4000 K are put into one group while planets around stars with effective temperature above 4000 K are put into another group.
	
	For each planet in the Sullivan et al. catalogue, a similar planet, in radius and period, are selected from the NASA archive. The ratio of radius and period between the two planets are calculated and multiplied with the Sullivan planets radius and period. This results in that the two planets are identical in radius and period. These ratios are then applied to the rest of the planets in this selected Kepler system to create an artificial system of planets.  
	
	The mass of the planet is required to simulate the system and is approximated using equation \ref{eq:mass_p_low} and \ref{eq:mass_p_high} obtained from \cite{2015ApJ...809...77S}:
	\begin{equation}
	\label{eq:mass_p_low}
	M_p = M_{\oplus} \left[0.440 \left(\frac{R_p}{R_{\oplus}}\right)^3 + 0.614\left(\frac{R_p}{R_{\oplus}}\right)^4\right]
	\end{equation}
	for planets with $R_p < 1,5 \; \mathrm{R_{\oplus}}$ where $R_{\oplus}$ is the radius of Earth and $M_{\oplus}$ is the mass of Earth. For planets with $R_p \geq 1,5$ the equation changes to:
	\begin{equation}
	\label{eq:mass_p_high}
	M_p = 2.69 M_{\oplus}\left(\frac{R_p}{R_{\oplus}}\right)^{0,93}
	\end{equation}
	
	When setting up the system the mean anomaly is required. This specifies the positions of all planet in a given system at a snapshot in time and is used as a starting point for the simulation. It is acquired from the number of transits and the orbital period of the planet by using a reference point specified when a planet is directly in front of the host star as seen from an observers point of view. This reference point corresponds to a mean anomaly of 90$^{\circ}$ at some time $T_i$, where $i$ is the number of the planet in the system. For a two planet system this means that $M_1=90^{\circ}$ at some time $T_1$ and $M_2=90^{\circ}$ at some time $T_2$. In order to calculate the mean anomaly at some time a time reference point is defined as $t=0$ and the goal is the calculate the mean anomaly of some planet $i$. This can be done by using the mean anomaly at $t = T_i$ and subtracting the number of degrees, $\xi$, the planet have traveled since then:
\begin{equation}
	M_i(t=0) = M_i(t=T_i) - \xi
\end{equation}
	$M_i(t=T_i) = 90^{\circ}$ and the number of degrees traveled is $360^{\circ}$ multiplied by the number of orbits since $t_0$:
\begin{equation}
	M_i(t=0) = 90 - 360 \frac{T_{epoch}}{P_i}
\end{equation}
	where $T_{epoch}$ is the transit epoch and $P_i$ is the period of the planet.
	
	TTVFast also requires inclination, eccentricity, longitude of the ascending node and argument of periapsis. These cannot be simply obtained and need to be assumed. For simplicity the inclination is assumed to be $90^{\circ}$ for all planets which results in the longitude of the ascending node to be 0. The eccentricity is obtained from a Rayleigh distribution with mode $\sigma = 0,03$ in order to not get eccentricities above 0.1 as that leads to unstable systems. The argument of periapsis is obtained from a uniform distribution where $0 < \omega < 360$. 
	
	How long a system will be observed depends on where in the sky it is located. This time can be obtained from the Web TESS Target tool \footnote{\url{https://heasarc.gsfc.nasa.gov/cgi-bin/tess/webtess/wtm.py}} if the right ascension, RA, and declination, dec, are known. This tool accepts csv files with RA and dec of the systems and outputs the number of times TESS will observe that system. Currently it only works with systems in the southern hemisphere i.e with dec below 0. Because of this our systems in the northern hemisphere have their dec flipped and are such placed in the souther hemisphere when the data is uploaded to the TESS tool. The dec is then flipped back before the data is used. 
	

\section{TTVFast}
\label{TTVFast_method}
	TTVFast is a program created by \cite{2014ApJ...787..132D} which simulates planetary systems using an n-body integrator. It requires information about the system in the form of:
	\begin{itemize}
		\item Gravitational constant in $AU^3 \mathrm{day}^{-2}M_{\odot}^{-1}$
		\item Mass of the star 
	\end{itemize}
	And also for each planet in the system:
	\begin{itemize}
		\item Period in days 
		\item Eccentricity
		\item Inclination 
		\item Longitude of ascending node
		\item Argument of periapsis 
		\item Mean anomaly at the reference time 
	\end{itemize}
	Where longitude of ascending node and argument of periapsis are orbital elements. The reference time is the time of the start of the integration, in this paper: $t_{ref}=0$. The program also requires parameters regarding the integration which are given in a setup file:
	\begin{itemize}
		\item Path to file containing info regarding the planets in the system.
		\item Reference time
		\item Time step which is $1/20$ of the period
		\item Final time which in this paper is the duration of the integration
		\item Number of planets
		\item Input flag which specifies in which coordinate system the input parameters are given. This paper uses Jacobi coordinates which relate to a input flag = 0.
	\end{itemize}
	With all these known the system can be simulated and the output are given as a number of times when a transit occurred and the number of the transiting planet.
\section{Simulation of Ofir objects}
	A paper written by \cite{2018ApJS..234....9O} used observed objects to obtain TTV signals. In order to determine the precision of the methods used in this paper, the systems observed by Ofir et al. are simulated by the same methods as in this paper for 4 years. The resulting amplitudes of the TTV signals are compared to those of the Ofir systems.
\section{Analyzing results from TTVFast}
	When all systems have been simulated with TTVFast the data need to be analyzed to find the TTV signals. The transit times need to be corrected as TTVFast outputs the time that a transit occurs. This is done by fitting a linear fit to the times and subtracting this fit. In order to easier see the amplitude of the TTV signals the times are corrected by subtracting the average time from every value which moves the middle of the graph to $y=0$. In order to obtain the amplitude of the TTVs the average of the maximum and minimum transit time are calculated. This is used as the amplitude and the corrected transit times are plotted.
	
	The position in the sky of the objects are of interest and are obtained from the Sullivan et al. catalogue. The positions are given in RA and dec and plotted in a sky map. These coordinates are given in the equatorial frame and converted to the ecliptic frame. When plotted, this produces sky maps which are colour coded according to the TTV amplitude and the multiplicity, i.e. the number of planets, of the systems.
	

\section{Error estimation}
	The error estimation used in this paper comes from equation 3 in \cite{2005Sci...307.1288H}:
	\begin{equation}
		\sigma_t \approx \left[\left(\Gamma t_T\right)^{-1/2}  \left(\frac{R_p}{R_{\star}}\right)^{-3/2}\right] t_T
	\end{equation}
	where $\Gamma$ is the photon count rate of the observed star, $t_T$ is the transit duration, $R_p$ and $R_{\star}$ is the radius of the planet and star in solar radii. For Kepler the photon count is $\Gamma = 7.8 \times 10^8\; 10^{-4(V-14)} \; \mathrm{hr^{-1}}$ where $V$ is the apparent magnitude of the star. From \cite{2015ApJ...809...77S} the photon flux for TESS at magnitude $I_c=0$ is $\Phi \approx 1.4 \times 10^6 \; \mathrm{s^{-1} cm^{-2}}$ and $\Phi \approx 1.4 \times 10^6 \; 10^{-0.4I_c} \; \mathrm{s^{-1} cm^{-2}}$ when $I_c \neq 0$. The diameter of a camera on TESS is 100 mm which gives an area of $A_{camera} = \pi (50 \; \mathrm{mm})^2 = 7859.98 \; \mathrm{mm}^2 = 78.54 \; \mathrm{cm}^2$. The photon count is the photon flux multiplied with the area of the camera:
	\begin{equation}
		\Gamma = \Phi A_{camera} = 1.4 \times 10^6\; 10^{-0.4I_c} \; \mathrm{s^{-1} cm^{-2}} \times 78.54 \; \mathrm{cm}^2 = 
	\end{equation}
	\begin{equation}
	1.10 \times 10^{10}\; 10^{-0.4I_c} \; \mathrm{s}^{-1} = 3.96 \times 10^{13} 10^{-0.4I_c}
	\end{equation}
\section{Stability simulations}
	The artificial systems that are created need to be checked for stability to determine if the assumptions and method of creating them results in realistic and stable systems. For this Rebound \citep{2012A&A...537A.128R}, more specifically the WHFast integrator \citep{2015MNRAS.452..376R}, is used. WHFast is a Wisdom-Holman integrator used for long duration planetary system simulations. For some systems close-encounters occur, WHFast is not a good integrator for these and therefore if such a system is found it will instead be simulated with IAS15 \citep{2015MNRAS.446.1424R}. IAS15 is another integrator for gravitational dynamics and although slower than WHFast it handles close-encounters better.
	
	WHFast requires the semi-major axis of each planet in each system. This is obtained from Kepler's law of periods:
	\begin{equation}
		a^3 = \frac{T^2 G M_{\star}}{4\pi^2} \Rightarrow a = \sqrt[3]{\frac{T^2 G M_{\star}}{4\pi^2}}
	\end{equation}
		where $a$ is the semi-major axis in AU and T is the period in years. 
		
	As the stability simulations take a long time not all systems can be checked for stability. The first systems are selected by their multiplicity in order to see that at least one system of each multiplicity is stable. The Hill radius is then calculated for each system:
	\begin{equation}
		r_{Hill} = \frac{a_1 + a_2}{2}\left(\frac{M_1 + M_2}{3M_{\star}}\right)^{1/3}
	\end{equation}
	where $a_{1,2}$ is the semi-major axis of the two inner most planets, $M_{1,2}$ is the mass of the two planets and $M_{\star}$ is the mass of the host star. This radius is then used to calculate the separation:
	\begin{equation}
		\Delta = \frac{a_2 - a_1}{r_{Hill}}
	\end{equation}
	The systems with lowest separation is the systems most probable to be unstable \citep{1996Icar..119..261C} and thus they are simulated for $10^5$ to $10^6$ years depending on the system, to ensure stability. 
	

	

\chapter{Results}

\iffalse 

\begin{verbatim}

       PROGRAM myfortran

       IMPLICIT NONE

       REAL*8 mag(20)
       REAL flux(20)
       INTEGER nstar

       WRITE(*,*) "This program calculates a magnitude"
       READ(*,*) flux(1)
       mag(1)=-2.5*LOG(flux(1))

\end{verbatim}

\fi




\section{Simulated TESS objects}
	The planetary radius of the artificial systems are plotted against the orbital periods in figure \ref{fig:RP_plot_temp_multi} where the multiplicity of the system are represented with colour and shape. Most systems consists of two or three planets and systems containing five or six planets are few. Almost all planets are larger than earth and have a far shorter period. Larger planets are easier to discover as they block out more of the light from the star and short period planets allows for more opportunities to observe the transit which results in that most discovered planets are large with short periods.

\begin{figure}[h!]
 	 \centering
 	 \includegraphics[width=\textwidth]{img/R_P-plot_numP1.png}
 	  	 \caption{Radius distribution as a function of period for the simulated TESS objects where the colour and shape corresponds to the multiplicity of the system.}
 	  	 \label{fig:RP_plot_temp_multi}
\end{figure}
	\newpage Figure \ref{fig:skymap_TESS} shows a sky map in the ecliptic frame of the simulated TESS objects and the number of times they are observed. The green area shows the approximate range where CHEOPS are able to observe the objects.  As expected, the systems in the ecliptic poles are observed the most times and most objects in the CHEOPS range are only observed once or not at all.

\begin{figure}[h!]
	\centering
	\includegraphics[width=\textwidth]{img/skymap_TESS_multi.png}
	  \caption{Position of each observed objects in the ecliptic frame, colour-coded to show the number of times the object is observed.}	
	  \label{fig:skymap_TESS}	
\end{figure}\newpage
\section{Simulations of Ofir systems}
	Figure \ref{fig:ampl_ofir} shows the distribution of TTV amplitudes from the objects from the Ofir catalogue. In this histogram, all systems with amplitude below 1 minute are filtered out as they are not of interest in this paper. It is clear that many systems show a small TTV signal but there are a significant portion of systems which show a higher amplitude. The error bars for each bin are approximated to be $\epsilon = \sqrt{N}$ where $N$ is the number of planets in the bin.
\begin{figure}
 	 \centering
	  \includegraphics[width=\textwidth]{img/ampl_ofir_66.png}
	  \caption{TTV amplitudes of the simulated objects from the Ofir catalogue.}
	 \label{fig:ampl_ofir}
\end{figure}  


\section{TTV signals from TESS objects}
	Figure \ref{fig:TTV1} shows an example of a TTV signal for a system at one of the ecliptic poles where the variation is clearly visible. The zero level corresponds to the average transit time in order to see the variations more clearly. The distribution of TTV amplitudes are found in figure \ref{fig:ampl_histo} where amplitudes below 1 minute are filtered out as otherwise they dominate the histogram. It is easy to see that low or no TTV signals dominate but there are some planets showing substantial TTV signals. The location in the sky of these objects are shown in figure \ref{fig:skymap_amp}. Many planets at the ecliptic poles show some form of TTV signal while many planets not in the poles show TTV signals they are generally weaker with a few exceptions. In the range that CHEOPS are able to observe about 1/4 of systems have a planet showing a detectable TTV signal.
\begin{figure}[h]
 	 \centering
	  \includegraphics[width=\textwidth]{img/232_1.png}
	  \caption{Observed-Calculated transit time as a function of time. The zero level corresponds to average transit time.}
	 \label{fig:TTV1}
\end{figure}
\begin{figure}[h!]
 	 \centering
	  \includegraphics[width=\textwidth]{img/ampl_tess_400.png}
	  \caption{Distribution of TTV amplitudes from the simulated TESS objects.}
	 \label{fig:ampl_histo}
\end{figure}
\begin{figure}[h!]
 	 \centering
	  \includegraphics[width=\textwidth]{img/skymap_TESS_amp.png}
	  \caption{Position in the sky in the ecliptic frame and the TTV amplitude.}
	 \label{fig:skymap_amp}
\end{figure}
\section{Error analysis}
		Figure \ref{fig:amp_error} shows the amplitude of the systems as a function of the error of the amplitude. Also shown is a line where y=x where the amplitude is equal to the error. The systems with an amplitude lower than the error cannot be considered as the signal can be due to noise.
\begin{figure}
 	 \centering
	  \includegraphics[width=\textwidth]{img/ampErrorLog.png}
	  \caption{Amplitude of the TTV signal as a function of the error where y=x is marked with a black line.}
	 \label{fig:amp_error}
\end{figure}
\section{Stability simulations}
	Figure \ref{fig:stability_sim} shows two examples of the results from the stability simulations. For all stability tested systems most  have not shown any instability within $10^5 / 10^6$ years. A few systems did eject a planet which  probably is the result of an close encounter. These systems were then re-simulated using IAS15 which showed that they were in fact stable and the ejection was a result from WHFast not being able to handle the close encounters. It can be noted that all of the planets are very close to the host star. This is a result from how the semi-major axis is calculated, as most planets in this paper have a short period the semi-major axis is small which results in very closely packed systems.

\begin{figure}
\centering
\begin{minipage}{.5\textwidth}
  \centering
  \includegraphics[width=1\linewidth]{img/180.jpg}
 

\end{minipage}%
\begin{minipage}{.5\textwidth}
  \centering
  \includegraphics[width=1\linewidth]{img/320.jpg}
  

\end{minipage}
\caption{Orbits of the planets in a three planet system (left panel) and a five planet system (right panel)}
\label{fig:stability_sim}
\end{figure}





\iffalse
\begin{table}[!h]
\caption{Example table from template}\smallskip
\label{table:1}
\centering  
\begin{tabular}{lrrc}
\hline\hline  
\smallskip
Id of star & I &  V & Var.? \\
\hline
1234 & 15.6 & 17.3 & No \\
5677 & 13.4 & 12.3 & Yes\\
\hline
\end{tabular}
\end{table}
\fi

\chapter{Discussion}
	These results may be used to give an approximation of how many systems in a given sample will show TTV signals. The systems can be observed in more detail by the Characterising Exoplanets Satellite, CHEOPS, or the James Webb Space telescope, JWST. CHEOPS are not able to measure stars at the poles and are limited to the sky around the plane of Earth's orbit which are marked as a green area in figure \ref{fig:skymap_TESS} and \ref{fig:skymap_amp}. As TESS will have coverage of the poles for the whole year many of the systems showing TTV signals will be located at the poles. The JWST on the other hand will be able to look at the poles. This makes James Webb a very good option for further studies of the TESS objects. 
	
	The artificial systems created in this paper contain some planets which show strong TTV signals, mainly positioned in the ecliptic poles as the systems positioned here are observed for a whole year which gives many more opportunities to detect TTVs. The shape of the histogram in figure \ref{fig:ampl_ofir} are the same as in the Ofir paper. This shows that the methods of simulating systems used in this paper are viable and give reasonable results.
	
	The way of determining the amplitude of the TTVs are very simplistic. It is simply an average of the highest and lowest value. This can be improved by some kind of fit as the TTV signals are often in the shape of a sine curve but due to time constraints this paper does not include this way of determining the amplitude.
	
	Even though these results show that CHEOPS might be able to find a planet showing TTV signals in every fourth system the real number is probably lower. An approximation used in this paper is that the inclination is $90^{\circ}$ for all planets. This creates systems where all planets are transiting which is not the case for real systems. There are many cases where none of the planets around a star are transiting and the planet that might transit might not show any TTV signal while other planets around the star might do.
	
	Using WHFast, most systems tested for stability show no sign of instability within $10^5 / 10^6$ years. The few that do are simulated using IAS15 where they are shown to be stable and the instability from WHFast can be considered an affect of WHFast's inability to handle close encounters.
	
	The error calculations seem to give reasonable results. As seen in figure \ref{fig:amp_error} many systems shows a TTV amplitude higher than the error but many systems shows a TTV amplitude lower than the error, for these systems no conclusion can be made as the signal could be true but it might as well be noise.
\chapter{Conclusions}
	This paper simulated TESS objects using data from the NASA exoplanet archive and results from the paper by \cite{2015ApJ...809...77S} on a search to find TTV signals caused by multiple planets in the systems. The results can be seen below:
	\begin{itemize}
		\item Most of the systems showing TTV signals are positioned at the ecliptic poles. This comes from the fact that TESS have continuous coverage of the poles during the year which increases the chance to see TTVs.
		\item In the range where CHEOPS are able to observe about 1/4 of every system will have a planet showing a detectable TTV signal.
	\end{itemize}

\section*{Acknowledgements}

Simulations in this paper made use of the REBOUND code which can be downloaded freely at \url{http://github.com/hannorein/rebound}.\vspace{0.5cm}\\
This research made use of Astropy, a community-developed core Python package for Astronomy (Astropy Collaboration, 2013).

%\bibliographystyle{natbib}
%\begin{thebibliography}{99}
%\bibitem[Alexander \& Armitage(2007)]{2007MNRAS.375..500A}
%  Alexander, R.~D., \& Armitage, P.~J. 2007, \mnras, 375, 500
%\bibitem[Santos et~al.(2001)Santos, Israelian, \& Mayor]{2001A&A...373.1019S}
%  Santos, N.~C., Israelian, G., \& Mayor, M. 2001, \aap, 373, 1019
%\end{thebibliography}

\bibliographystyle{aa}
\bibliography{references}

\begin{appendix}

\chapter{This is an appendix}
\label{ap:input_code}
You can put long mathematical derivations or tables in appendices.

\chapter{This is another appendix}

\end{appendix}

\end{document}
